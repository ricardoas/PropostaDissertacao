% Universidade Federal de Campina Grande
% Proposta de Dissertação de Mestrado em Ciencia da Computacao
%
% Aluno: Ricardo Araújo Santos - ricardo@lsd.ufcg.edu.br - ricardo@dsc.ufcg.edu.br
% Orientadora: Raquel Vigolvino Lopes
% Dezembro de 2010

\documentclass[a4paper,titlepage,12pt]{article}

%language
\usepackage[brazil]{babel}
\usepackage[utf8]{inputenc} % codificar arquivo em UTF-8
\usepackage[T1]{fontenc}

%tabelas
\usepackage{longtable}
\usepackage{colortbl}

\usepackage{color}
\usepackage{times}
\usepackage{fancyheadings}
\usepackage{fancyvrb}

\usepackage{algorithmic}
\usepackage[nothing]{algorithm}
\usepackage{latexsym}

\usepackage{graphicx,url}

\usepackage[portuguese]{nomencl}


\sloppy

% Estilo e espacamento ----------------------------------------
\newlength{\defbaselineskip}
\setlength{\defbaselineskip}{\baselineskip}
\newcommand{\setlinespacing}[1]
           {\setlength{\baselineskip}{#1 \defbaselineskip}}

\setcounter{topnumber}{2}
\renewcommand{\topfraction}{.7}
\setcounter{bottomnumber}{1}
\renewcommand{\bottomfraction}{.3}
\setcounter{totalnumber}{3}
\renewcommand{\textfraction}{.2}
\renewcommand{\floatpagefraction}{.5}
\setcounter{dbltopnumber}{2}
\renewcommand{\dbltopfraction}{.7}
\renewcommand{\dblfloatpagefraction}{.5}

\oddsidemargin -28pt
\evensidemargin -28pt
\marginparwidth 50pt
\marginparsep 5pt
\topmargin -27pt
\hoffset 15mm
\textheight 237mm
\textwidth 155mm
\renewcommand{\baselinestretch}{1.5}
% ------------------------------------------------------------------------

% Tradução dos comandos do algorithmic ----------------------------------------
% \renewcommand{\algorithmicend}{\textbf{fim}}
% \renewcommand{\algorithmicif}{\textbf{se}}
% \renewcommand{\algorithmicthen}{\textbf{ent\~{a}o}}
% \renewcommand{\algorithmicelse}{\textbf{sen\~{a}o}}
% \renewcommand{\algorithmicendif}{\textbf{fim-se}}
% \renewcommand{\algorithmicelsif}{\algorithmicelse\ \algorithmicif}
% \renewcommand{\algorithmicendif}{\algorithmicend\ \algorithmicif}
% \renewcommand{\algorithmicfor}{\textbf{para}}
% \renewcommand{\algorithmicforall}{\textbf{paratodos}}
% \renewcommand{\algorithmicdo}{\textbf{fa\c{c}a}}
% \renewcommand{\algorithmicendfor}{\algorithmicend\ \algorithmicfor}
% \renewcommand{\algorithmicwhile}{\textbf{enquanto}}
% \renewcommand{\algorithmicendwhile}{\algorithmicend\ \algorithmicwhile}
% \renewcommand{\algorithmicloop}{\textbf{loop}}
% \renewcommand{\algorithmicendloop}{\algorithmicend\ \algorithmicloop}
% \renewcommand{\algorithmicrepeat}{\textbf{repita}}
% \renewcommand{\algorithmicuntil}{\textbf{at\'{e}}}
% \floatname{algorithm}{Algoritmo}
% ------------------------------------------------------------------------

\renewcommand{\nomname}{Lista de Siglas e Abreviaturas}

\makenomenclature

\begin{document}

% Capa ------------------------------------------------------------------------

\pagestyle{empty}

\begin{center}
{\textbf{\Large \textsc{Universidade Federal de Campina Grande}}}
\end{center}

\begin{center}
\textbf{{\Large \textsc{Centro de Engenharia Elétrica e Informática}}}
\end{center}

\begin{center}
{\large \textsc{\textbf{Curso de Mestrado em Ciência da Computação}}}
\end{center}

~\\ \\

\begin{center}
{\LARGE \textsc{\textbf{Proposta de Dissertação de Mestrado}}}
\end{center}

~\\ \\

\begin{center}
{\Large \textsc{\textbf{<<Título da Proposta>>}}}
\end{center}

~\\ \\

\begin{center}
\textbf{\textsc{Mestrando(a)} \\
\textsc{Ricardo Araújo Santos}}
\end{center}

\begin{center}
\textbf{\textsc{Orientadora} \\
\textsc{Raquel Vigolvino Lopes}}
\end{center}

~\\ \\ \\

\begin{center}
\textbf{{\large \textsc{Campina Grande}}
\\
{\large \textsc{Dezembro - 2010}}}
\end{center}

\newpage
\cleardoublepage

% ------------------------------------------------------------------------

% Algarismos romanos para as páginas de listagem.
\pagestyle{plain}
\pagenumbering{roman}


\printnomenclature[1cm]
\newpage

\listoffigures
\newpage

\listoftables
\newpage

\tableofcontents
\newpage

% Corpo do documento -------------------

% Algarismos arábicos para as páginas do corpo.
\pagestyle{plain}
\setcounter{page}{1}
\pagenumbering{arabic}

%%%%%%%%%%%%%%%%%%%%%%%%%%%%%%%%%%%%%%%%%%%%%%%%%%%%%%%%%%%%%%%%%%%%%%%%%%%%%%%%
%% Definicao do cabecalho: secao do lado esquerdo e numero da pagina do lado direito
\pagestyle{fancy}
\addtolength{\headwidth}{\marginparsep}\addtolength{\headwidth}{\marginparwidth}\headwidth
= \textwidth
\renewcommand{\sectionmark}[1]{\markright{\thesection\ #1}}\lhead[\fancyplain{}{\bfseries\thepage}]%
         {\fancyplain{}{\emph{\rightmark}}}\rhead[\fancyplain{}{\bfseries\leftmark}]%
             {\fancyplain{}{\bfseries\thepage}}\cfoot{}

%%%%%%%%%%%%%%%%%%%%%%%%%%%%%%%%%%%%%%%%%%%%%%%%%%%%%%%%%%%%%%%%%%%%%%%%%%%%%%%%


\section{Introdução}
\label{sec:introducao}

Na introduo voc deve contextualizar o problema que estar estudando. Em particular, voc deve descrever brevemente a rea na qual estar trabalhando e introduzir de forma clara os problemas da rea relevantes ao seu trabalho. Observe que a seo no fala de solues mas de problemas existindo na rea de interesse.
Sem contar apndices e referncias bibliogrficas, uma proposta de dissertao de mestrado geralmente contm uma dezena de pginas.

\section{Objetivo da Proposta}
\label{sec:objetivo}

Nesta seo, voc deve falar da soluo que voc prope aos problemas apresentados anteriormente. O enfoque  sobre "o qu" voc vai fazer.

\section{Relev\^{a}ncia da Proposta}
\label{sec:relev}

Aqui, o enfoque  o "porqu". Voc pode se concentrar em responder  seguinte pergunta: "Se meu trabalho for bem sucedido, o que ter mudado na rea sob estudo?". Em outras palavras, quais so as contribuies planejadas?

\section{Metodologia de Trabalho}
\label{sec:metodologia}

Aqui o enfoque  o "como".  Quais so os passos que devero ser desenvolvidos, e em que ordem, para que o trabalho seja feito. Uma tabela semelhante  Tabela~\ref{tab:atividades} pode ser utilizada para apoiar a apresentao.

\section{Cronograma}
\label{cronograma}

%%\usepackage{colortbl} incluir para colorir as células.

Pretende-se realizar as atividades enumeradas na seção~\ref{atividades} segundo o cronograma representado na tabela~\ref{table:cronograma}: 

\begin{table}[h]
\centering
\begin{tabular}{|c|c|c|c|c|c|c|c|c|c|c|c|c|c|c|}
    \hline
 & Jan & Fev & Mar & Abr & Mai & Jun & Jul & Ago & Set & Out & Nov & Dez & Jan & Fev\\    
	\cline{2-15}
 & & & & & & & & & & & & & & \\
	\hline
A1 & \cellcolor{black} & \cellcolor{black} & & & & & & & & & & & & \\
	\hline
A2 & & \cellcolor{black} & \cellcolor{black} & & & & & & & & & & & \\
	\hline
A3 & & & & \cellcolor{black} & & & & & & & & & & \\
	\hline
A4 & & & & & \cellcolor{black} & & & & & & & & & \\
	\hline
A5 & & & & & & \cellcolor{black} & \cellcolor{black} & \cellcolor{black} & \cellcolor{black} & \cellcolor{black} & & & & \\
    \hline
A6 & & & & & & & & & \cellcolor{black} & & & & & \\
    \hline
A7 & & & \cellcolor{black} & \cellcolor{black} & \cellcolor{black} & \cellcolor{black} & \cellcolor{black} & \cellcolor{black} & \cellcolor{black} & \cellcolor{black} & \cellcolor{black} & \cellcolor{black} & & \\
    \hline
A8 & & & & & \cellcolor{black} & \cellcolor{black} & \cellcolor{black} & \cellcolor{black} & \cellcolor{black} & \cellcolor{black} & \cellcolor{black} & \cellcolor{black} & \cellcolor{black}& \cellcolor{black} \\
    \hline
A9 & & & & & & & & & & & & & \cellcolor{black} & \cellcolor{black} \\
	\hline
\multicolumn{1}{c}{\rule{0.3cm}{0cm}} &
\multicolumn{1}{c}{\rule{0.4cm}{0cm}} &
\multicolumn{1}{c}{\rule{0.4cm}{0cm}} &
\multicolumn{1}{c}{\rule{0.4cm}{0cm}} &
\multicolumn{1}{c}{\rule{0.4cm}{0cm}} &
\multicolumn{1}{c}{\rule{0.4cm}{0cm}} &
\multicolumn{1}{c}{\rule{0.4cm}{0cm}} &
\multicolumn{1}{c}{\rule{0.4cm}{0cm}} &
\multicolumn{1}{c}{\rule{0.4cm}{0cm}} &
\multicolumn{1}{c}{\rule{0.4cm}{0cm}} &
\multicolumn{1}{c}{\rule{0.4cm}{0cm}} &
\multicolumn{1}{c}{\rule{0.4cm}{0cm}} &
\multicolumn{1}{c}{\rule{0.4cm}{0cm}} &
\multicolumn{1}{c}{\rule{0.4cm}{0cm}} &
\multicolumn{1}{c}{\rule{0.4cm}{0cm}} \\
\end{tabular}
\caption{Cronograma}
\label{table:cronograma}
\end{table}


% 
% \begin{table}[h]
% \caption{Atividades planejadas.}
% \label{tab:atividades}
%  \small
%   \begin{center}
%     \begin{tabular}{|c||p{13cm}|} \hline
%       {\bf Atividade}   &{\bf Descrio} \\\hline
%       {\bf 1}   & Realizar uma pesquisa bibliogrfica sobre as solues existentes para o problema de \textit{xpto}.  \\\hline
%       {\bf 2}   & Elaborar relat\'{o}rio t\'{e}cnico contendo a avalia\c{c}\~{a}o das solu\c{c}\~{o}es encontradas. \\\hline
%       {\bf 3}   & Levantar os requisitos bsicos da soluo desejada. \\\hline
%       {\bf 3.1}   & Definir um esboo de uma soluo a ser implementada que atenda aos requisitos levantados. \\\hline
%       {\bf 4}   & Elaborar o projeto da soluo e definir a linguagem de programao a ser usada para implementar a ferramenta. O processo de desenvolvimento utilizado ser baseado no Processo Unificado, utilizando UML como linguagem de modelagem. \\\hline
%       {\bf 4.1}   & Fazer o levantamento detalhado de requisitos funcionais da ferramenta, gerando um modelo de Anlise. \\\hline
%       {\bf 4.2}   & Definir a arquitetura da ferramenta, gerando um modelo arquitetural. \\\hline
%       {\bf 4.3}   & Definir a linguagem de programao na qual a ferramenta ser implementada. \\\hline
%       {\bf 4.4}   & Elaborar o projeto da ferramenta, gerando um modelo de projeto. \\\hline
%       {\bf 5}  & Implementar a ferramenta utilizando a linguagem de programao escolhida. Sero utilizados testes de unidade durante esta fase. \\\hline
%       {\bf 6}  & Elaborar um artigo contendo a avaliao da ferramenta. \\\hline
%       {\bf 7}  & Elaborar a redao da dissertao de mestrado. \\\hline
%       {\bf 8}  & Defender a disserta\c{c}\~{a}o de mestrado. \\\hline
%     \end{tabular}
%   \end{center}
% %  \normalsize
% \end{table}

\newpage


\section{Cronograma}
\label{sec:cronograma}

O cronograma apresenta a dimenso "quando". As atividades mencionadas na Metodologia devem ser cronogramadas, incluindo data de incio, data final, dependncias entre atividades.
Um exemplo de cronograma  apresentado na Tabela~\ref{tab:cronograma}.

\begin{table}[ht]
\caption{Cronograma do projeto de pesquisa.} \label{tab:cronograma}
%  \scriptsize
  \begin{center}
    \begin{tabular}{|c|c||c|c|c|c|c|c|c|c|c|c|c|c|c|c|} \hline
                       & & \multicolumn{14}{c|}{ {\bf Atividade } } \\\hline \hline
      {\bf Ano}  &{\bf Ms}  &{\bf 1}  &{\bf 2}  &{\bf 3} &{\bf 4}  &{\bf 5}  &{\bf 6}  &{\bf 7} &{\bf 8}  &{\bf 9}  &{\bf 10} &{\bf 11}  &{\bf 12}  &{\bf 13}  &{\bf 14}\\\hline
      {\bf 2005} &{\bf Jun}  & X       & X       &        &         &         &         &        &         &         &         &          &          &          &        \\\hline
      {\bf 2005} &{\bf Jul}  & X       &         & X      &         &         &         &        &         &         &         &          &          &          &        \\\hline
      {\bf 2005} &{\bf Ago}  & X       &         &        & X       &         &         &        &         &         &         &          &          &          &        \\\hline
      {\bf 2005} &{\bf Set}  & X       &         &        &         & X       &         &        &         &         &         &          &          &          &        \\\hline
      {\bf 2005} &{\bf Out}  & X       &         &        &         & X       &         &        &         &         &         &          &          &          &        \\\hline
      {\bf 2005} &{\bf Nov}  & X       &         &        &         & X       &         &        &         &         &         &          &          &          &        \\\hline
      {\bf 2005} &{\bf Dez}  & X       &         &        &         &         & X       &        &         &         &         &          &          &          &        \\\hline
      {\bf 2006} &{\bf Jan}  & X       &         &        &         &         &         & X      &         &         &         &          &          &          &        \\\hline
      {\bf 2006} &{\bf Fev}  & X       &         &        &         &         &         & X      &         &         &         &          &          &          &        \\\hline
      {\bf 2006} &{\bf Mar}  &         &         &        &         &         &         &        & X       & X       &         &          &          &          &        \\\hline
      {\bf 2006} &{\bf Abr}  &         &         &        &         &         &         &        &         &         & X       &          &          &          &        \\\hline
      {\bf 2006} &{\bf Mai}  &         &         &        &         &         &         &        &         &         & X       &          &          &          &        \\\hline
      {\bf 2006} &{\bf Jun}  &         &         &        &         &         &         &        &         &         &         & X        &          &          &        \\\hline
      {\bf 2006} &{\bf Jul}  &         &         &        &         &         &         &        &         &         &         &          & X        &          &        \\\hline
      {\bf 2006} &{\bf Ago}  &         &         &        &         &         &         &        &         &         &         &          &          &  X       &        \\\hline
      {\bf 2006} &{\bf Set}  &         &         &        &         &         &         &        &         &         &         &          &          &  X       &        \\\hline
      {\bf 2006} &{\bf Out}  &         &         &        &         &         &         &        &         &         &         &          &          &  X       &        \\\hline
      {\bf 2006} &{\bf Nov}  &         &         &        &         &         &         &        &         &         &         &          &          &  X       &        \\\hline
      {\bf 2006} &{\bf Dez}  &         &         &        &         &         &         &        &         &         &         &          &          &          & X      \\\hline
    \end{tabular}
  \end{center}
%  \normalsize
\end{table}

\newpage

%% Bibliografia ----------------------------------------------------------------
\bibliographystyle{alpha} % estilo de bibliografia
\bibliography{proposta} % arquivos com as entradas bib.

\end{document}
